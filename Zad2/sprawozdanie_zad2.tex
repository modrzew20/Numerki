\documentclass{classrep}
\usepackage[utf8]{inputenc}
\frenchspacing

\usepackage{graphicx}
\usepackage[usenames,dvipsnames]{color}
\usepackage[hidelinks]{hyperref}

\usepackage{amsmath, amssymb, mathtools, icomma}

\usepackage{fancyhdr, lastpage}
\pagestyle{fancyplain}
\fancyhf{}
\renewcommand{\headrulewidth}{0pt}
\cfoot{\thepage\ / \pageref*{LastPage}}


\studycycle{Informatyka, studia dzienne, I st.}
\coursesemester{IV}

\coursename{Inteligentna Analiza Danych}
\courseyear{2019/2020}

\courseteacher{mgr inż. Paweł Tarasiuk}
\coursegroup{środa, 14:30}

\author{%
  \studentinfo[229963@edu.p.lodz.pl]{Daniel Modrzejewski}{229963}\\
  \studentinfo[230004@edu.p.lodz.pl]{Mateusz Srebnik}{230004}%
}

\title{Zadanie 2.: metoda iteracyjna Jacobiego (iteracji prostej) }

\sloppy

\begin{document}
\maketitle
\thispagestyle{fancyplain}

\section{Cel}
{\color{black}
Celem zadania drugiego jest zaimplementowanie programu mającego na celu  rozwiązywania układu N równań liniowych z N niewiadomymi za pomocą metody iteracyjnej Jacobiego (iteracji prostej).
}

\section{Wprowadzenie}
{\color{blue}
We wprowadzeniu należy zaprezentować niezbędną teorię potrzebną do realizacji
zadania (przy czym należy tu ograniczyć się wyłącznie do tego, co było
wykorzystane), tak aby inny student, który nigdy wcześniej nie zetknął się z tą
tematyką, potrafił zrozumieć dalszy opis. Część ta powinna wprowadzać wszystkie
wykorzystywane wzory, oznaczenia itp., do których należy się odwoływać w
dalszej części niniejszego sprawozdania. Zamieszczony tu własny opis teorii
(a nie skopiowany!) należy poprzeć odwołaniami bibliograficznymi do literatury
zamieszczonej na końcu.}

\section{Opis implementacji}
{\color{blue}
Należy tu zamieścić krótki i zwięzły opis zaprojektowanych klas oraz powiązań
między nimi. Często powinien się tu również znaleźć diagram UML (diagram klas)
prezentujący najistotniejsze elementy stworzonej aplikacji. Należy także podać,
w jakim języku programowania została stworzona aplikacja.}

\section{Materiały i metody}
{\color{blue}
W tym miejscu należy opisać, jak przeprowadzone zostały wszystkie badania,
których wyniki i dyskusja zamieszczane są w dalszych sekcjach. Opis ten
powinien być na tyle dokładny, aby osoba czytająca go potrafiła wszystkie
przeprowadzone badania samodzielnie powtórzyć w celu zweryfikowania ich
poprawności. Przy opisie należy odwoływać się i stosować do
opisanych w sekcji drugiej wzorów i oznaczeń, a także w jasny sposób opisać
cel konkretnego testu. Najlepiej byłoby wyraźnie wyszczególnić (ponumerować)
poszczególne eksperymenty tak, aby łatwo było się do nich odwoływać dalej.}

\section{Wyniki}
{\color{blue}
W tej sekcji należy zaprezentować, dla każdego przeprowadzonego eksperymentu,
kompletny zestaw wyników w postaci tabel, wykresów (preferowane) itp. Powinny
być one tak ponazywane, aby było wiadomo, do czego się odnoszą. Wszystkie
tabele i wykresy należy oczywiście opisać (opisać co jest na osiach, w
kolumnach itd.) stosując się do przyjętych wcześniej oznaczeń. Nie należy tu
komentować i interpretować wyników, gdyż miejsce na to jest w kolejnej sekcji.
Tu również dobrze jest wprowadzić oznaczenia (tabel, wykresów), aby móc się do
nich odwoływać poniżej.}

\section{Dyskusja}
{\color{blue}
Sekcja ta powinna zawierać dokładną interpretację uzyskanych wyników
eksperymentów wraz ze szczegółowymi wnioskami z nich płynącymi. Najcenniejsze
są, rzecz jasna, wnioski o charakterze uniwersalnym, które mogą być istotne
przy innych, podobnych zadaniach. Należy również omówić i wyjaśnić wszystkie
napotkane problemy (jeśli takie były). Każdy wniosek powinien mieć poparcie we
wcześniej przeprowadzonych eksperymentach (odwołania do konkretnych wyników).
Jest to jedna z najważniejszych sekcji tego sprawozdania, gdyż prezentuje
poziom zrozumienia badanego problemu.}

\section{Wnioski}
{\color{blue}
W tej, przedostatniej, sekcji należy zamieścić podsumowanie najważniejszych
wniosków z sekcji poprzedniej. Najlepiej jest je po prostu wypunktować. Znów,
tak jak poprzednio, najistotniejsze są wnioski o charakterze uniwersalnym.}

\begin{thebibliography}{0}
  \bibitem{l2short} T. Oetiker, H. Partl, I. Hyna, E. Schlegl.
    \textsl{Nie za krótkie wprowadzenie do systemu \LaTeX2e}, 2007, dostępny
    online. \url{https://ctan.org/tex-archive/info/lshort/polish/lshort2e.pdf}.
\end{thebibliography}

{\color{blue}
Na końcu należy obowiązkowo podać cytowaną w sprawozdaniu literaturę, z której
grupa korzystała w trakcie prac nad zadaniem.}

\end{document}
