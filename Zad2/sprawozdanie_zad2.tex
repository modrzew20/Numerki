\documentclass{classrep}
\usepackage[utf8]{inputenc}
\frenchspacing

\usepackage{graphicx}
\usepackage[usenames,dvipsnames]{color}
\usepackage[hidelinks]{hyperref}

\usepackage{amsmath, amssymb, mathtools, icomma}

\usepackage{fancyhdr, lastpage}
\pagestyle{fancyplain}
\fancyhf{}
\renewcommand{\headrulewidth}{0pt}
\cfoot{\thepage\ / \pageref*{LastPage}}


\studycycle{Informatyka, studia dzienne, I st.}
\coursesemester{IV}

\coursename{Inteligentna Analiza Danych}
\courseyear{2019/2020}

\courseteacher{mgr inż. Paweł Tarasiuk}
\coursegroup{środa, 14:30}

\author{%
  \studentinfo[229963@edu.p.lodz.pl]{Daniel Modrzejewski}{229963}\\
  \studentinfo[230004@edu.p.lodz.pl]{Mateusz Srebnik}{230004}%
}

\title{Zadanie 2.: metoda iteracyjna Jacobiego (iteracji prostej) }

\sloppy

\begin{document}
\maketitle
\thispagestyle{fancyplain}

\section{Cel}
{\color{black}
Celem zadania drugiego jest zaimplementowanie programu mającego na celu  rozwiązywania układu N równań liniowych z N niewiadomymi za pomocą metody iteracyjnej Jacobiego (iteracji prostej).
}

\section{Wprowadzenie}
{

Metoda pozwala na obliczenie układu N równań liniowych z N niewiadomymi. Przybliżonym wynikiem układu równań bedzie wektor x . Na poczatku wektor bedzie zawierał same 0, z każdą nastepna iteracja algorytmu Jackobiego bedzie on przybliżał sie do prawidłowego wyniku. W tym celu korzystamy ze wzoru \n

x^{n+1} = Mx^{n} +NB

gdzie 

x^{n+1} - wynik iteracji algorytmu 
x^{n} - wynik poprzedniej iteracji algorytmu 

A - to macierz współczynników Ax=B
B - to macierz wyników rownań
N - D^{-1} , gdzie D jest macierzą skladajaca sie z elementów głównej przekatnej macierzy A

M - to wynik -N(L+U), gdzie (L+U) to macierz A z zastapionymi wartosciami 0 na głowej przekątnej \n

mozemy zapisac, że \n
A = L + U + D
}

\section{Opis implementacji}
{
Program został napisany w jezyku Python, składa sie z dwóch klas. Klasa main odpowiada za backendowa część programu , a klasa gui na frontendowa.
Klasa main zawiera w sobie funkcje:
- sprawdzająca czy podana macierz ma odpowiedni rozmiar oraz czy ma 0 na przekatnych,
- zmiana kolejnosci wierszy
- sprawdzenie czy macierz jest przekatnie dominujaca 
- mnożenie macierzy
- algorytm 
}

\section{Materiały i metody}
{
Badania zostały przeprowadzone dla 5 macierzy , 3 przekątnie dominujących oraz 2 nie przekątnie dominujacych (nieoznaczonej i sprzecznej)
Badania:
1. Macierz przekątnie dominująca 
$$
\mathbf{X} =
\left| \begin{array}{ccc}
0 & 1 &  0 \\
0 & 0 &  1 \\
1 & 0 &  0 \\
\end{array} \right|
$$




W tym miejscu należy opisać, jak przeprowadzone zostały wszystkie badania,
których wyniki i dyskusja zamieszczane są w dalszych sekcjach. Opis ten
powinien być na tyle dokładny, aby osoba czytająca go potrafiła wszystkie
przeprowadzone badania samodzielnie powtórzyć w celu zweryfikowania ich
poprawności. Przy opisie należy odwoływać się i stosować do
opisanych w sekcji drugiej wzorów i oznaczeń, a także w jasny sposób opisać
cel konkretnego testu. Najlepiej byłoby wyraźnie wyszczególnić (ponumerować)
poszczególne eksperymenty tak, aby łatwo było się do nich odwoływać dalej.}

\section{Wyniki}
{\color{blue}
W tej sekcji należy zaprezentować, dla każdego przeprowadzonego eksperymentu,
kompletny zestaw wyników w postaci tabel, wykresów (preferowane) itp. Powinny
być one tak ponazywane, aby było wiadomo, do czego się odnoszą. Wszystkie
tabele i wykresy należy oczywiście opisać (opisać co jest na osiach, w
kolumnach itd.) stosując się do przyjętych wcześniej oznaczeń. Nie należy tu
komentować i interpretować wyników, gdyż miejsce na to jest w kolejnej sekcji.
Tu również dobrze jest wprowadzić oznaczenia (tabel, wykresów), aby móc się do
nich odwoływać poniżej.}

\section{Dyskusja}
{\color{blue}
Sekcja ta powinna zawierać dokładną interpretację uzyskanych wyników
eksperymentów wraz ze szczegółowymi wnioskami z nich płynącymi. Najcenniejsze
są, rzecz jasna, wnioski o charakterze uniwersalnym, które mogą być istotne
przy innych, podobnych zadaniach. Należy również omówić i wyjaśnić wszystkie
napotkane problemy (jeśli takie były). Każdy wniosek powinien mieć poparcie we
wcześniej przeprowadzonych eksperymentach (odwołania do konkretnych wyników).
Jest to jedna z najważniejszych sekcji tego sprawozdania, gdyż prezentuje
poziom zrozumienia badanego problemu.}

\section{Wnioski}
{\color{blue}
W tej, przedostatniej, sekcji należy zamieścić podsumowanie najważniejszych
wniosków z sekcji poprzedniej. Najlepiej jest je po prostu wypunktować. Znów,
tak jak poprzednio, najistotniejsze są wnioski o charakterze uniwersalnym.}

\begin{thebibliography}{0}
  \bibitem{l2short} T. Oetiker, H. Partl, I. Hyna, E. Schlegl.
    \textsl{Nie za krótkie wprowadzenie do systemu \LaTeX2e}, 2007, dostępny
    online. \url{https://ctan.org/tex-archive/info/lshort/polish/lshort2e.pdf}.
\end{thebibliography}

{
https://college.cengage.com/mathematics/larson/elementary_linear/5e/students/ch08-10/chap_10_2.pdf
http://www.algorytm.org/procedury-numeryczne/metoda-jacobiego.html
}

\end{document}
